\chapter{Cadre du projet et étude de l'existant}\label{chap1}


\lhead{chapitre 1: Cadre du projet et étude de l'existant}
\dominitoc 
\rhead{\thepage}
\minitoc
\section{Introduction}\label{chap1:intro}
Ce premier chapitre établit le cadre contextuel et organisationnel de la recherche en présentant de manière systématique l'entreprise d'accueil BACOVET, filiale du groupe BACOSPORT et acteur significatif de l'industrie textile tunisienne. L'analyse s'attache à mettre en évidence les caractéristiques organisationnelles, le positionnement stratégique au sein de la chaîne de valeur textile, ainsi que l'orientation de l'entreprise vers l'innovation technologique.

Dans un second temps, le projet est contextualisé dans le paradigme de l'Industrie 4.0, permettant d'identifier de manière structurée les défis opérationnels liés à la planification et au suivi de production. Une étude critique et approfondie de l'existant, fondée sur l'observation empirique et l'analyse documentaire, permettra de formuler précisément la problématique à résoudre.

Enfin, une introduction structurée de la solution intelligente proposée et de la méthodologie de recherche adoptée clôturera ce chapitre, établissant ainsi les fondements théoriques et pratiques du travail de recherche entrepris.

\section{Présentation de l'entreprise d'accueil : Bacovet}\label{Chap1:sect1}

\subsection{Historique et identité} 
Le groupe BACOSPORT, fondé en 1967, constitue aujourd'hui un acteur majeur et structurant de l'industrie textile tunisienne, spécialisé dans la confection de vêtements sportswear, de sous-vêtements, de pyjamas et de maillots de bain. Dans l'écosystème organisationnel du groupe, BACOVET, filiale stratégique implantée à Boumerdes, occupe une position clé dans la chaine de valeur en assurant les opérations critiques de coupe industrielle, de préparation des tissus, de sérigraphie, de contrôle qualité et de logistique de transfert vers l'atelier de confection. L'expertise technique reconnue et la rigueur organisationnelle déployée par BACOVET contribuent de manière substantielle à la compétitivité internationale du groupe et à sa capacité à répondre aux exigences qualitatives et temporelles des marchés export.

\begin{figure}[H]
  \centering
  \begin{minipage}[b]{0.45\textwidth}
    \centering
    \includegraphics[width=\textwidth]{Chapitre1/images/2.jpg}
    \caption{logo d'entreprise}
    \label{form}
  \end{minipage}
  \hfill
  \begin{minipage}[b]{0.50\textwidth}
    \centering
    \includegraphics[width=\textwidth]{Chapitre1/images/1.jpg }
    \caption{Siège de l'entreprise Bacsport}
      \label{inscri}

  \end{minipage}
\end{figure}


\subsection{Partenariats stratégiques et présence mondiale}
Le succès du groupe Bacosport et de sa filiale Bacovet repose sur une collaboration étroite avec de grandes marques internationales du secteur textile.
Grâce à son savoir-faire technique, sa flexibilité et sa capacité à répondre à des exigences de qualité élevées, Bacovet entretient des partenariats durables avec des enseignes renommées à travers l'Europe et le bassin méditerranéen.
Parmi ses principaux clients figurent des marques telles que Décathlon, La Redoute, Damart, Sunflair, DD, Romy Aim, et Calao.
Ces collaborations stratégiques témoignent de la confiance des donneurs d'ordre internationaux et renforcent la position de Bacovet comme un acteur de référence dans le textile tunisien à vocation exportatrice.

En s'inscrivant dans des chaînes d'approvisionnement mondiales, Bacovet adopte des standards de qualité et de traçabilité conformes aux attentes des marchés européens. Cette ouverture internationale pousse également l'entreprise à investir dans la digitalisation et dans des solutions innovantes, pour maintenir un niveau de performance concurrentiel.


\begin{figure}[H]
  \centering
  \begin{minipage}[b]{0.45\textwidth}
    \centering
    \includegraphics[width=\textwidth]{Chapitre1/images/3.png}
    \caption{Clients de BACOVET}
    \label{form}
  \end{minipage}
  \end{figure}
  
 
\subsection{Certifications de Bacovet}

Bacovet s'engage à garantir la qualité et la sécurité de ses produits à travers le respect de normes internationales reconnues. L'entreprise est certifiée selon la norme ISO 9001 : 2015, qui atteste de l'efficacité de son système de management de la qualité et de son orientation vers l'amélioration continue.

Dans le cadre de sa responsabilité sociétale et environnementale, Bacovet applique également les standards OEKO-TEX® Standard 100, qui garantissent que les tissus utilisés sont exempts de substances nocives et répondent aux exigences de sécurité pour la santé humaine.

Ces certifications renforcent la crédibilité de Bacovet auprès de ses partenaires et confirment son engagement envers la qualité, la durabilité et la conformité aux normes internationales.
\begin{figure}[H]
  \centering
  \begin{minipage}[b]{0.45\textwidth}
    \centering
    \includegraphics[width=\textwidth]{Chapitre1/images/4.png}
    \caption{ Certifications  de BACOVET}
    \label{form}
  \end{minipage}
   \hfill
  \begin{minipage}[b]{0.50\textwidth}
    \centering
    \includegraphics[width=\textwidth]{Chapitre1/images/5.png }
    
      \label{inscri}

  \end{minipage}
  \end{figure}
\section{Organisation et organigramme}
\begin{figure}[H]
  \centering
  \begin{minipage}[b]{0.45\textwidth}
    \centering
    \includegraphics[width=\textwidth]{Chapitre1/images/Organigramme.drawio.png}
    \caption{ organigramme de BACOVET}
    \label{form}
  \end{minipage}
  \end{figure}
 \section{Processus global chez BACOVET}
Chez BACOVET, la chaîne de production textile est structurée en plusieurs étapes successives assurant une traçabilité , une qualité constante et un respect  des délais.La figure\ref{ajouter}présente les processus global de production 
\begin{figure}[H]
\begin{center}

    \includegraphics[height=11.5cm]{Chapitre1/images/process-Page-3.drawio.png}
     \captionof{figure}{\emph{Diagramme de séquence:"les procédure global de production chez BACOVET  "}}
    \label{ajouter}
\end{center}
 \end{figure}
\subsection{Processus détaillé de l'atelier de coupe}
L'atelier de coupe joue un rôle essentiel dans la production textile chez BACOVET. C'est à ce niveau que les rouleaux de tissu, qui représentent la matière la plus chère, sont découpés en pièces précises. Une bonne coupe permet de limiter les pertes de tissu et de garantir que toutes les pièces ont les bonnes dimensions. Cet atelier se situe entre l'approvisionnement des matières et l'assemblage final. Une gestion efficace de cette étape aide à respecter les délais, à bien utiliser les ressources, et à assurer la qualité des produits. Le respect des procédures (du marquage à la vérification finale) permet de réduire les erreurs et d'améliorer la performance globale. La figure~\ref{fig:processus-atelier-coupe} présente le processus détaillé de l'atelier de coupe.

\begin{figure}[H]
\begin{center}
    \includegraphics[width=0.95\textwidth,height=0.85\textheight,keepaspectratio]{Chapitre1/images/process-Page-4.drawio.png}
     \captionof{figure}{\emph{Diagramme de séquence : Processus détaillé de l'atelier de coupe}}
    \label{fig:processus-atelier-coupe}
\end{center}
\end{figure}

\begin{enumerate}
    \item \textbf{Réception des ordres de fabrication (OF)}
    
L'atelier reçoit des ordres de fabrication (OF) qui indiquent les modèles à produire, les quantités et les tissus à utiliser. Ces ordres sont envoyés depuis la planification. On vérifie que toutes les informations sont correctes et que les rouleaux de tissu sont disponibles.

\item \textbf{Relaxation du tissu (si nécessaire)}

Certains tissus doivent se détendre avant d'être découpés, pour éviter qu'ils ne rétrécissent plus tard. Cette étape se fait en laissant reposer le tissu à l'air libre ou à la vapeur pendant 24 à 72 heures.

\item \textbf{Préparation des rouleaux et découpe du papier}

Avant la coupe, les rouleaux de tissu sont préparés. On vérifie leur longueur, leur largeur et leur couleur. On coupe le papier de base (papier matelas) qui sera placé sous le tissu pour la coupe.

\item \textbf{Matelassage}

Le tissu est empilé en plusieurs couches sur la table de coupe. Le nombre de plis est défini selon l'épaisseur du tissu et le volume de production.

\begin{figure}[H]
\begin{center}
    \includegraphics[height=5cm]{Chapitre1/images/ZONE matellas.jpg}
     \captionof{figure}{\emph{Zone matelassage}}
    \label{fig:zone-matelassage}
\end{center}
\end{figure}

\item \textbf{Placement et marquage}

L'étape de placement et de marquage est essentielle pour bien préparer la coupe. Le placement consiste à organiser les patrons sur le tissu de façon à utiliser le moins de matière possible, tout en respectant le sens du tissu, le droit fil et les motifs. Le marquage sert à tracer les contours des pièces et à indiquer les repères nécessaires au montage.

\item \textbf{Coupe}

La coupe peut se faire de différentes manières selon le type de production. Pour les petites séries, on utilise la coupe manuelle avec des ciseaux. Pour les plus grandes séries, la coupe mécanique est préférée, avec des machines à lame verticale. La coupe automatisée permet une grande précision grâce à la programmation des machines.

\begin{figure}[H]
  \centering
  \begin{minipage}[b]{0.45\textwidth}
    \centering
    \includegraphics[width=\textwidth]{Chapitre1/images/8.jpg}
    \caption{Coupe automatisée}
    \label{fig:coupe-auto}
  \end{minipage}
   \hfill
  \begin{minipage}[b]{0.50\textwidth}
    \centering
    \includegraphics[width=\textwidth]{Chapitre1/images/9.jpg}
    \caption{Machine de coupe}
    \label{fig:machine-coupe}
  \end{minipage}
\end{figure}

\item \textbf{Déchargement des pièces coupées}

Les pièces découpées sont triées et empilées avec soin. Cette étape requiert une vigilance particulière afin d'éviter les mélanges ou les endommagements.

\item \textbf{Sérigraphie / Heat Transfert (si nécessaire)}

Certaines pièces nécessitent un marquage par sérigraphie ou transfert thermique. Cette étape permet l'identification ou la décoration des produits.

\begin{figure}[H]
  \centering
  \begin{minipage}[b]{0.60\textwidth}
    \centering
    \includegraphics[width=\textwidth]{Chapitre1/images/16.jpg}
    \caption{Machine Sérigraphie}
    \label{fig:serigraphie}
  \end{minipage}
\end{figure}

\item \textbf{Préparation et contrôle des vignettes}

Des vignettes contenant les informations essentielles (taille, modèle, OF) sont générées et associées aux pièces. Leur exactitude est cruciale pour le bon déroulement des étapes suivantes.

\item \textbf{Départage}

Le départage consiste à trier les pièces selon les tailles et les modèles. Il s'agit d'une opération de précision qui facilite le travail de l'atelier de confection.

\item \textbf{Contrôle qualité final}

Avant expédition, chaque lot est inspecté pour valider sa conformité. Cette étape est cruciale pour garantir un niveau de qualité élevé.

\begin{figure}[H]
  \centering
  \begin{minipage}[b]{0.50\textwidth}
    \centering
    \includegraphics[width=\textwidth]{Chapitre1/images/14.jpg}
    \caption{Contrôle Qualité}
    \label{fig:controle-qualite}
  \end{minipage}
\end{figure}

\item \textbf{Expédition vers l'atelier de confection}

Les pièces sont conditionnées puis transférées à l'atelier de confection. Cette étape est suivie informatiquement via G.Pro, assurant une traçabilité complète.

\begin{figure}[H]
  \centering
  \begin{minipage}[b]{0.50\textwidth}
    \centering
    \includegraphics[width=\textwidth]{Chapitre1/images/13.jpg}
    \caption{Atelier de confection}
    \label{fig:atelier-confection}
  \end{minipage}
\end{figure}
\end{enumerate}

\subsection{Personnel de l'atelier de coupe}
L'atelier de coupe mobilise une équipe de quarante-quatre personnes organisées selon une structure hiérarchique claire :

\begin{itemize}
    \item \textbf{Un responsable d'atelier} : assure la supervision générale et la planification des opérations
    \item \textbf{Deux chefs d'équipe} : encadrent les opérateurs et suivent la production quotidienne
    \item \textbf{Vingt opérateurs de matelassage} : préparent et empilent les tissus selon les spécifications
    \item \textbf{Six opérateurs de coupe} : découpent les pièces et assurent la maintenance de premier niveau des machines
    \item \textbf{Treize opératrices de départage} : trient et classent les pièces coupées par taille et modèle
    \item \textbf{Deux contrôleurs qualité} : vérifient la conformité des lots et assurent la traçabilité
\end{itemize}

L'outil central de coordination est le \textit{dispatch sheet}, document de planification quotidienne. Il synthétise les ordres de fabrication en cours, les quantités à produire, les priorités et les délais. Chaque matin, le responsable distribue les dispatch sheets aux chefs d'équipe, qui répartissent les tâches entre les opérateurs. Cette pratique garantit une visibilité claire sur les objectifs et facilite le suivi de l'avancement, contribuant à la réduction des temps d'attente et à l'optimisation des ressources.

\subsection{Matériels et paramètres techniques}\label{sec:parametres-techniques-atelier}
L'atelier dispose d'équipements adaptés aux différents volumes de production. Les tables de coupe (15 à 20 mètres) permettent le matelassage de plusieurs dizaines de plis. Les machines se déclinent en trois catégories : outils manuels pour les petites séries, machines à lame verticale pour les volumes moyens, et systèmes automatisés à commande numérique pour les grandes séries, intégrant des logiciels de placement optimisé.

Le flux logistique s'organise autour de zones fonctionnelles : réception et relaxation (conditions contrôlées), matelassage, coupe, départage et conditionnement. Des chariots élévateurs et transpalettes facilitent les déplacements entre zones.

Le tableau~\ref{tab:parametres-atelier} présente les paramètres techniques détaillés de l'atelier de coupe.

\begin{table}[H]
\centering
\begin{adjustbox}{max width=\textwidth}
\begin{tabular}{|l|l|}
\hline
\textbf{Élément} & \textbf{Valeur} \\
\hline
Nombre de tables & 6 (dont 1 automatique, 4 manuelles, 1 vide) \\
\hline
Chariot matelasseur automatique & 1 \\
\hline
Robots de coupe & 2 (translation horizontale sur 5 tables) \\
\hline
Effectif sur les tables & 17 opérateurs \\
\hline
Équipe de départage & 13 postes / 13 personnes \\
\hline
Zones de stock & 3 (avant/après sérigraphie) \\
\hline
\end{tabular}
\end{adjustbox}
\caption{Paramètres techniques de l'atelier de coupe}
\label{tab:parametres-atelier}
\end{table}

\section{Cadre du projet}\label{chap1:cadre}

L'atelier de coupe de BACOVET, bien que performant, souffre de l'absence d'un système digitalisé de planification et de suivi en temps réel. Cette lacune génère des pertes de temps, un manque de visibilité sur la production et une capacité limitée d'anticipation des retards. Ce projet vise à développer une solution numérique intelligente intégrant les principes de l'Industrie 4.0 pour optimiser la gestion de l'atelier.

\subsection{Visualisation de la situation actuelle de l'atelier de coupe}

Afin de mieux comprendre le contexte opérationnel et les limites du système actuel, la Figure~\ref{fig:carte_mentale_atelier_chap1} présente la carte mentale de la situation existante de l'atelier de coupe de BACOVET, réalisée à l'aide de l'outil Coggle. Cette représentation synthétise les flux d'information et les ruptures digitales entre les différents outils utilisés (Divatex et G.Pro).

On observe notamment :

\begin{itemize}
    \item \textbf{Une fragmentation du système d'information} entre Divatex (gestion amont) et G.Pro (suivi aval) sans interconnexion.
    \item \textbf{Une zone grise manuelle} au niveau des étapes de matelassage et coupe, où aucun suivi numérique ni mesure automatique des temps n'est effectué.
    \item \textbf{Une absence de visibilité globale} : les temps réels, les retards et les performances ne sont pas mesurés, entraînant un manque d'anticipation et de réactivité.
\end{itemize}

\begin{figure}[H]
\centering
\includegraphics[angle=90,width=0.40\textheight,keepaspectratio]{Chapitre2/images/23.png}
\caption{Carte mentale de la situation actuelle de l'atelier de coupe}
\label{fig:carte_mentale_atelier_chap1}
\end{figure}

\subsection{Définition du projet (Méthode SIPOC)}

La méthode SIPOC permet d'identifier les éléments clés du projet et de clarifier son périmètre d'intervention.

\begin{table}[H]
\centering
\captionsetup{justification=centering}
\caption{Analyse SIPOC du projet}
\label{tab:sipoc}
\begin{tabular}{|p{1.35cm}|p{4.45cm}|}
\hline
\rowcolor{gray!30}
\textbf{Élément} & \textbf{Description} \\
\hline
\textbf{Suppliers} & Direction BACOVET, service informatique, opérateurs de l'atelier, fournisseurs de technologies IA \\
\hline
\textbf{Inputs} & Données historiques de production, spécifications techniques des machines, contraintes opérationnelles, cahier des charges \\
\hline
\textbf{Process} & Diagnostic DMAIC, collecte et préparation des données, développement d'algorithmes IA, conception du tableau de bord, tests et déploiement \\
\hline
\textbf{Outputs} & Modèle prédictif IA, tableau de bord en temps réel, système d'ordonnancement, documentation technique, KPIs \\
\hline
\textbf{Customers} & Responsable d'atelier, chefs d'équipe, opérateurs, direction générale, service qualité \\
\hline
\end{tabular}
\end{table}

\subsection{Objectifs du projet}

\textbf{Objectif général :} Concevoir et déployer une solution numérique intelligente pour optimiser la gestion de l'atelier de coupe, intégrant l'IA et les principes de l'Industrie 4.0.

\textbf{Objectifs spécifiques :}
\begin{enumerate}
    \item Réaliser un diagnostic complet de l'atelier selon la méthodologie DMAIC
    \item Développer un modèle IA de prédiction des temps de matelassage (précision > 85\%)
    \item Concevoir un système d'ordonnancement intelligent optimisant l'allocation des ressources
    \item Créer un tableau de bord de pilotage en temps réel avec KPIs pertinents
    \item Valider et déployer la solution avec formation des utilisateurs
\end{enumerate}

\subsection{Périmètre du projet (Méthode QQOQCCP)}

La méthode QQOQCCP permet de définir précisément le périmètre d'intervention du projet.

\begin{table}[htbp]
\centering
\small
\caption{Analyse QQOQCCP du projet}
\label{tab:qqoqccp}
\begin{adjustbox}{max width=\textwidth}
\begin{tabular}{|l|p{10cm}|}
\hline
\rowcolor{gray!30}
\textbf{Question} & \textbf{Réponse} \\
\hline
\textbf{Qui ?} & Étudiant en Master Génie Industriel, encadrement académique (ENIM) et professionnel (BACOVET), utilisateurs finaux (chefs d'équipe, opérateurs) \\
\hline
\textbf{Quoi ?} & Transformation digitale de l'atelier de coupe : modèle prédictif IA, système d'ordonnancement, tableau de bord en temps réel, documentation technique \\
\hline
\textbf{Où ?} & Atelier de coupe de BACOVET, Boumerdes, Tunisie. \textbf{Exclusions :} ateliers de sérigraphie, confection et contrôle qualité \\
\hline
\textbf{Quand ?} & 4 mois (février-mai 2025) : Diagnostic (1 mois), Développement (1,5 mois), Tests et déploiement (1 mois), Formation et clôture (0,5 mois) \\
\hline
\textbf{Comment ?} & Méthodologies : DMAIC (diagnostic), CRISP-ML(Q) (développement IA), Agile/Scrum. Technologies : Python, Machine Learning, FastAPI, React, PostgreSQL \\
\hline
\textbf{Combien ?} & 1 étudiant à temps plein, encadrement hebdomadaire, collaboration avec 3 opérateurs et 2 chefs d'équipe, accès aux données historiques (2 ans) \\
\hline
\textbf{Pourquoi ?} & \textbf{Problèmes :} absence de système digitalisé, pertes de temps, manque de visibilité. \textbf{Bénéfices attendus :} réduction des temps d'attente (20-30\%), amélioration du taux d'utilisation des machines (15-25\%), visibilité en temps réel \\
\hline
\end{tabular}
\end{adjustbox}
\end{table}

\subsection{Limites et contraintes}

\begin{itemize}
    \item \textbf{Périmètre limité :} Atelier de coupe uniquement (matelassage et coupe)
    \item \textbf{Durée :} 4 mois imposant une approche MVP (Minimum Viable Product)
    \item \textbf{Intégration :} Respect de l'infrastructure informatique existante de BACOVET
    \item \textbf{Données :} Qualité et complétude des données historiques à vérifier
    \item \textbf{Adoption :} Accompagnement au changement et formation nécessaires
\end{itemize}

Cette définition structurée du cadre du projet, appuyée sur les méthodes SIPOC et QQOQCCP, permet d'établir une base solide pour la conduite du projet et garantit une compréhension partagée entre toutes les parties prenantes. Elle constitue également un référentiel pour l'évaluation de l'atteinte des objectifs et la mesure de la performance du projet.

\section{Conclusion du chapitre}\label{chap1:conclusion}
Ce chapitre a établi le cadre contextuel et organisationnel de la recherche en présentant l'entreprise d'accueil BACOVET et son positionnement stratégique dans l'écosystème textile tunisien. L'analyse critique de l'existant a permis d'identifier les limitations structurelles de l'atelier de coupe, notamment l'absence d'un système digitalisé intégré. 

Ce projet de transformation digitale constitue une opportunité stratégique majeure pour BACOVET de renforcer son agilité industrielle et sa compétitivité opérationnelle, tout en posant les fondations méthodologiques et technologiques d'une digitalisation progressive et évolutive des autres maillons de la chaîne de production. Les résultats attendus de cette recherche contribueront à la fois au renforcement des capacités opérationnelles de l'entreprise et à l'enrichissement des connaissances académiques sur l'application des principes de l'Industrie 4.0 au secteur textile tunisien.
