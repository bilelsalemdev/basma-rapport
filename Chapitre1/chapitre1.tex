\chapter{Cadre du projet et étude de l'existant}\label{chap1}


\lhead{chapitre 1: Cadre du projet et étude de l'existant}
\dominitoc 
\rhead{\thepage}
\minitoc
\section{Introduction}\label{chap1:intro}
Ce premier chapitre établit le cadre contextuel et organisationnel de la recherche en présentant de manière systématique l'entreprise d'accueil BACOVET, filiale du groupe BACOSPORT et acteur significatif de l'industrie textile tunisienne. L'analyse s'attache à mettre en évidence les caractéristiques organisationnelles, le positionnement stratégique au sein de la chaîne de valeur textile, ainsi que l'orientation de l'entreprise vers l'innovation technologique.

Dans un second temps, le projet est contextualisé dans le paradigme de l'Industrie 4.0, permettant d'identifier de manière structurée les défis opérationnels liés à la planification et au suivi de production. Une étude critique et approfondie de l'existant, fondée sur l'observation empirique et l'analyse documentaire, permettra de formuler précisément la problématique à résoudre.

Enfin, une introduction structurée de la solution intelligente proposée et de la méthodologie de recherche adoptée clôturera ce chapitre, établissant ainsi les fondements théoriques et pratiques du travail de recherche entrepris.

\section{Présentation de l'entreprise d'accueil : Bacovet}\label{Chap1:sect1}

\subsection{Historique et identité} 
Le groupe BACOSPORT, fondé en 1967, constitue aujourd'hui un acteur majeur et structurant de l'industrie textile tunisienne, spécialisé dans la confection de vêtements sportswear, de sous-vêtements, de pyjamas et de maillots de bain. Dans l'écosystème organisationnel du groupe, BACOVET, filiale stratégique implantée à Boumerdes, occupe une position clé dans la chaîne de valeur en assurant les opérations critiques de coupe industrielle, de préparation des tissus, de sérigraphie, de contrôle qualité et de logistique de transfert vers l'atelier de confection. L'expertise technique reconnue et la rigueur organisationnelle déployée par BACOVET contribuent de manière substantielle à la compétitivité internationale du groupe et à sa capacité à répondre aux exigences qualitatives et temporelles des marchés export.

\begin{figure}[H]
  \centering
  \begin{minipage}[b]{0.45\textwidth}
    \centering
    \includegraphics[width=\textwidth]{Chapitre1/images/2.jpg}
    \caption{logo d'entreprise}
    \label{form}
  \end{minipage}
  \hfill
  \begin{minipage}[b]{0.50\textwidth}
    \centering
    \includegraphics[width=\textwidth]{Chapitre1/images/1.jpg }
    \caption{Siège de l'entreprise Bacsport}
      \label{inscri}

  \end{minipage}
\end{figure}


\subsection{Partenariats stratégiques et présence mondiale}
Le succès du groupe Bacosport et de sa filiale Bacovet repose sur une collaboration étroite avec de grandes marques internationales du secteur textile.
Grâce à son savoir-faire technique, sa flexibilité et sa capacité à répondre à des exigences de qualité élevées, Bacovet entretient des partenariats durables avec des enseignes renommées à travers l'Europe et le bassin méditerranéen.
Parmi ses principaux clients figurent des marques telles que Décathlon, La Redoute, Damart, Sunflair, DD, Romy Aim, et Calao.
Ces collaborations stratégiques témoignent de la confiance des donneurs d'ordre internationaux et renforcent la position de Bacovet comme un acteur de référence dans le textile tunisien à vocation exportatrice.

En s'inscrivant dans des chaînes d'approvisionnement mondiales, Bacovet adopte des standards de qualité et de traçabilité conformes aux attentes des marchés européens. Cette ouverture internationale pousse également l'entreprise à investir dans la digitalisation et dans des solutions innovantes, pour maintenir un niveau de performance concurrentiel.


\begin{figure}[H]
  \centering
  \begin{minipage}[b]{0.45\textwidth}
    \centering
    \includegraphics[width=\textwidth]{Chapitre1/images/3.png}
    \caption{Clients de BACOVET}
    \label{form}
  \end{minipage}
  \end{figure}
  
 
\subsection{Certifications de Bacovet}

Bacovet s'engage à garantir la qualité et la sécurité de ses produits à travers le respect de normes internationales reconnues. L'entreprise est certifiée selon la norme ISO 9001 : 2015, qui atteste de l'efficacité de son système de management de la qualité et de son orientation vers l'amélioration continue.

Dans le cadre de sa responsabilité sociétale et environnementale, Bacovet applique également les standards OEKO-TEX® Standard 100, qui garantissent que les tissus utilisés sont exempts de substances nocives et répondent aux exigences de sécurité pour la santé humaine.

Ces certifications renforcent la crédibilité de Bacovet auprès de ses partenaires et confirment son engagement envers la qualité, la durabilité et la conformité aux normes internationales.
\begin{figure}[H]
  \centering
  \begin{minipage}[b]{0.45\textwidth}
    \centering
    \includegraphics[width=\textwidth]{Chapitre1/images/4.png}
    \caption{ Certifications  de BACOVET}
    \label{form}
  \end{minipage}
   \hfill
  \begin{minipage}[b]{0.50\textwidth}
    \centering
    \includegraphics[width=\textwidth]{Chapitre1/images/5.png }
    
      \label{inscri}

  \end{minipage}
  \end{figure}
\section{Organisation et organigramme}
\begin{figure}[H]
  \centering
  \begin{minipage}[b]{0.45\textwidth}
    \centering
    \includegraphics[width=\textwidth]{Chapitre1/images/Organigramme.drawio.png}
    \caption{ organigramme de BACOVET}
    \label{form}
  \end{minipage}
  \end{figure}
 \section{Processus global chez BACOVET}
Chez BACOVET, la chaîne de production textile est structurée en plusieurs étapes successives assurant une traçabilité , une qualité constante et un respect  des délais.La figure\ref{ajouter}présente les processus global de production 
\begin{figure}[H]
\begin{center}

    \includegraphics[height=11.5cm]{Chapitre1/images/process-Page-3.drawio.png}
     \captionof{figure}{\emph{Diagramme de séquence:"les procédure global de production chez BACOVET  "}}
    \label{ajouter}
\end{center}
 \end{figure}
 \subsection{Processus détaillé de l'atelier de coupe}
L'atelier de coupe joue un rôle essentiel dans la production textile chez BACOVET. C'est à ce niveau que les rouleaux de tissu, qui représentent la matière la plus chère, sont découpés en pièces précises. Une bonne coupe permet de limiter les pertes de tissu et de garantir que toutes les pièces ont les bonnes dimensions. Cet atelier se situe entre l'approvisionnement des matières et l'assemblage final. Une gestion efficace de cette étape aide à respecter les délais, à bien utiliser les ressources, et à assurer la qualité des produits. Le respect des procédures (du marquage à la vérification finale) permet de réduire les erreurs et d'améliorer la performance globale.La figure\ref{A}présente Processus détaillé de l'atelier de coupe
\begin{figure}[H]
\begin{center}

    \includegraphics[height=17cm]{Chapitre1/images/process-Page-4.drawio.png}
     \captionof{figure}{\emph{Diagramme de séquence:"Processus détaillé de l'atelier de coupe  "}}
    \label{A}
\end{center}
\end{figure}
\begin{enumerate}
    \item \textbf Réception des ordres de fabrication (OF)
    
 L'atelier reçoit des ordres de fabrication (OF) qui indiquent les modèles à produire, les quantités et les tissus à utiliser. Ces ordres sont envoyés depuis la planification. On vérifie que toutes les informations sont correctes et que les rouleaux de tissu sont disponibles. Ensuite, le travail est réparti entre les équipes.
Suivi : temps entre la réception de l'OF et le début de la coupe, vérification des données reçues.

Outils : lecteurs de codes-barres pour scanner les rouleaux.
\item \textbf Relaxation du tissu (si nécessaire)

Certains tissus doivent « se détendre » avant d'être découpés, pour éviter qu'ils ne rétrécissent plus tard. Cette étape se fait en laissant reposer le tissu à l'air libre ou à la vapeur pendant 24 à 72 heures.

Suivi : durée de relaxation, taux de retrait mesuré après repos.

Outils : capteurs de température et d'humidité dans la zone de repos.
\item \textbf Préparation des rouleaux et découpe du papier

Avant la coupe, les rouleaux de tissu sont préparés. On vérifie leur longueur, leur largeur et leur couleur  On coupe le papier de base (papier matelas) qui sera placé sous le tissu pour la coupe.
Suivi : temps de préparation, gaspillage de papier, précision des mesures.
\item \textbf Matelassage

Le tissu est empilé en plusieurs couches sur la table de coupe. Le nombre de plis est défini selon l'épaisseur du tissu et le volume de production.
\begin{figure}[H]
\begin{center}

    \includegraphics[height=5cm]{Chapitre1/images/ZONE matellas.jpg}
     \captionof{figure}{\emph{Zone matelassage}}
    \label{A}
\end{center}
\end{figure}

\item \textbf Placement et marquage

L'étape de placement et de marquage est essentielle pour bien préparer la coupe. Le placement consiste à organiser les patrons sur le tissu de façon à utiliser le moins de matière possible, tout en respectant le sens du tissu, le droit fil et les motifs (comme les rayures). Cela permet d'éviter le gaspillage et d'assurer un bon rendu final. Ensuite, le marquage sert à tracer les contours des pièces, à indiquer les repères nécessaires au montage (comme les crans ou les emplacements de poches), et à numéroter les pièces pour faciliter leur suivi. Une vérification finale est faite avant la découpe pour s'assurer que tout est bien positionné.
\item \textbf coupe

Avant de commencer la coupe, il faut préparer tous les outils nécessaires. Cela comprend la vérification de l'affûtage des lames, le bon fonctionnement des machines, ainsi que la préparation des accessoires comme les pinces, les règles ou les équerres. La coupe peut ensuite se faire de différentes manières selon le type de production. Pour les petites séries, on utilise souvent la coupe manuelle avec des ciseaux, en suivant les tracés avec précision, surtout dans les zones courbes, tout en respectant les tolérances. Pour les plus grandes séries, la coupe mécanique est préférée, avec des machines à lame verticale. Il faut alors régler la hauteur de coupe selon l'épaisseur du matelas et surveiller la qualité du travail tout au long du processus. Enfin, la coupe automatisée permet une grande précision grâce à la programmation des machines. Les données de placement sont chargées, et la coupe se fait automatiquement avec un contrôle qualité en temps réel.
\begin{figure}[H]
  \centering
  \begin{minipage}[b]{0.45\textwidth}
    \centering
    \includegraphics[width=\textwidth]{Chapitre1/images/8.jpg}
    \caption{coupe automatiseé }
    \label{form}
  \end{minipage}
   \hfill
  \begin{minipage}[b]{0.50\textwidth}
    \centering
    \includegraphics[width=\textwidth]{Chapitre1/images/9.jpg }
    
      \label{inscri}

  \end{minipage}
  \end{figure}
\item \textbf  Déchargement des Pièces Coupées

Les pièces découpées sont triées et empilées avec soin. Cette étape requiert une vigilance particulière afin d'éviter les mélanges ou les endommagements
\item \textbf  Sérigraphie / Heat Transfert (si nécessaire)

Certaines pièces nécessitent un marquage par sérigraphie ou transfert thermique. Cette étape permet l'identification ou la décoration des produits. Des capteurs contrôlent la température et la pression appliquées.
\begin{figure}[H]
  \centering
  \begin{minipage}[b]{0.60\textwidth}
    \centering
    \includegraphics[width=\textwidth]{Chapitre1/images/16.jpg}
    \caption{Machine Sérigraphie }
    \label{form}
  \end{minipage}
    \end{figure}
\item \textbf Préparation et Contrôle des Vignettes

Des vignettes contenant les informations essentielles (taille, modèle, OF) sont générées et associées aux pièces. Leur exactitude est cruciale pour le bon déroulement des étapes suivantes.
KPI : Taux d'exactitude des données.
\item \textbf Départage

Le départage consiste à trier les pièces selon les tailles et les modèles. Il s'agit d'une opération de précision qui facilite le travail de l'atelier de confection
\item \textbf  Contrôle Qualité Final

Avant expédition, chaque lot est inspecté pour valider sa conformité. Cette étape est cruciale pour garantir un niveau de qualité élevé
\begin{figure}[H]
  \centering
  \begin{minipage}[b]{0.50\textwidth}
    \centering
    \includegraphics[width=\textwidth]{Chapitre1/images/14.jpg}
    \caption{Contrôle Qualité }
    \label{form}
  \end{minipage}
    \end{figure}
\item \textbf  Expédition vers l'Atelier de Confection

Les pièces sont conditionnées puis transférées à l'atelier de confection. Cette étape est suivie informatiquement via G.Pro, assurant une traçabilité complète.
\begin{figure}[H]
  \centering
  \begin{minipage}[b]{0.50\textwidth}
    \centering
    \includegraphics[width=\textwidth]{Chapitre1/images/13.jpg}
    \caption{Atelier de confection }
    \label{form}
  \end{minipage}
    \end{figure}
\end{enumerate}
 \subsection{Durabilité et responsabilité environnementale}
Consciente des enjeux climatiques et environnementaux actuels, Bacosport, et par extension sa filiale Bacovet, s'inscrivent dans une démarche de production durable et responsable. L'entreprise considère la préservation de l'environnement comme un pilier fondamental de sa stratégie de développement industriel.

Dans cette optique, plusieurs initiatives écologiques majeures ont été mises en œuvre, notamment le projet photovoltaïque visant à transformer les sites de production en véritables sources d'énergie renouvelable. Cette transition énergétique permet non seulement de réduire significativement l'empreinte carbone de l'entreprise, mais également de renforcer son indépendance énergétique.

À ce jour, plus de 300 panneaux solaires ont été installés sur le site de Jammel, entraînant une réduction de 60 % de la dépendance énergétique du site. Fort de ce succès, le projet sera prochainement répliqué sur le site de Boumerdes, où est implantée Bacovet, consolidant ainsi la stratégie de transition énergétique du groupe.

L'ensemble de ces initiatives s'inscrit dans un engagement global de réduction de l'impact environnemental, tout en maintenant un haut niveau de qualité et de performance industrielle.
Par ce positionnement, Bacovet démontre qu'il est possible d'allier efficacité économique, innovation technologique et durabilité environnementale, en cohérence avec les objectifs de l'Industrie 4.0 verte.
\begin{figure}[H]
  \centering
  \begin{minipage}[b]{0.50\textwidth}
    \centering
    \includegraphics[width=\textwidth]{Chapitre1/images/17.jpg}
    \caption{le projet photovoltaïque }
    \label{form}
  \end{minipage}
  \end{figure}

\section{Cadre du projet}\label{chap1:cadre}
Dans un environnement industriel caractérisé par une intensification de la concurrence internationale et des exigences croissantes en matière de qualité, de réactivité, de traçabilité et d'optimisation des coûts, le secteur textile tunisien est confronté à l'impératif d'une évolution rapide et structurée vers des modes de production plus performants et plus flexibles.

Au sein de l'écosystème organisationnel du groupe BACOSPORT, la société BACOVET occupe une position stratégique déterminante dans la chaîne de valeur textile. L'entreprise regroupe plusieurs activités complémentaires et interdépendantes, notamment la coupe industrielle, la sérigraphie, la confection et le contrôle qualité, contribuant ainsi de manière substantielle à la performance globale du groupe sur les marchés nationaux et internationaux.

Cependant, l'analyse critique de l'existant révèle que l'atelier de coupe, bien que structuré et performant sur le plan opérationnel, présente des limitations structurelles significatives liées à l'absence d'un système digitalisé intégré d'ordonnancement, de suivi et de contrôle en temps réel. Cette lacune technologique et organisationnelle génère des pertes de temps opérationnelles non négligeables, une visibilité restreinte sur l'avancement des opérations, et une capacité limitée d'anticipation des retards ou de réactivité face aux imprévus de production.

Dans cette perspective, le présent projet de recherche s'inscrit dans une démarche structurée de transformation digitale ciblée, visant à intégrer de manière progressive et cohérente les principes fondamentaux de l'Industrie 4.0 au sein de l'atelier de coupe de BACOVET. 

L'objectif principal de cette recherche consiste à concevoir, développer et mettre en place une solution numérique intelligente, fondée sur l'intelligence artificielle et les technologies de l'information, permettant de réaliser les objectifs opérationnels suivants :
  \begin{itemize}
 \item Assurer un suivi en temps réel et une traçabilité complète des opérations, notamment au niveau des phases critiques de matelassage et de coupe

 \item Optimiser la gestion des plannings de production et l'allocation des ressources disponibles (machines, opérateurs)

 \item Garantir une traçabilité exhaustive des opérations et des flux de production

 \item Améliorer significativement les performances opérationnelles de l'atelier, mesurées par des indicateurs clés de performance (KPIs) quantifiables
    \end{itemize}

\section{Conclusion du chapitre}\label{chap1:conclusion}
Ce chapitre a établi le cadre contextuel et organisationnel de la recherche en présentant l'entreprise d'accueil BACOVET et son positionnement stratégique dans l'écosystème textile tunisien. L'analyse critique de l'existant a permis d'identifier les limitations structurelles de l'atelier de coupe, notamment l'absence d'un système digitalisé intégré. 

Ce projet de transformation digitale constitue une opportunité stratégique majeure pour BACOVET de renforcer son agilité industrielle et sa compétitivité opérationnelle, tout en posant les fondations méthodologiques et technologiques d'une digitalisation progressive et évolutive des autres maillons de la chaîne de production. Les résultats attendus de cette recherche contribueront à la fois au renforcement des capacités opérationnelles de l'entreprise et à l'enrichissement des connaissances académiques sur l'application des principes de l'Industrie 4.0 au secteur textile tunisien.
